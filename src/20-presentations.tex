\section{Section 1}
\subsection{Subsection 1}
\begin{frame}
  \frametitle{フレームの題名}
  \begin{itemize}
    \item ここにテキストを入力
    \item \textcite{_article}
    \item \textcite{_book}
    \item \textcite{_incollection}
    \item \textcite{_inproceedings,_misc,_techreport}
  \end{itemize}

  \note{
    ここにノートを入力
  }
\end{frame}


\begin{frame}
  \frametitle{フレーム2の題名}
  \begin{columns}
    \begin{column}{0.5\linewidth}
      \begin{align}
        e^{i\pi} + 1 = 0
      \end{align}
    \end{column}
    \begin{column}{0.5\linewidth}
      \begin{itemize}
        \item ここにテキストを入力
      \end{itemize}
    \end{column}
  \end{columns}

  \transduration<1>{2}%
  \transfade<1>%

  \note{
    ここにノート2を入力
  }
\end{frame}


\subsection{Subsection 1}
\begin{frame}
  \frametitle{フレーム3の題名}
  \begin{columns}
    \begin{column}{0.667\linewidth}
      \includegraphics[width=\linewidth]{example-image}
    \end{column}
    \begin{column}{0.333\linewidth}
      \begin{itemize}
        \item ここにテキストを入力
      \end{itemize}
    \end{column}
  \end{columns}

  \onslide<2>{%
    \begin{tikzpicture}[overlay,remember picture]
      \pgftransformshift{\pgfpointanchor{current page}{center}}%
      \node[%
        rectangle,%
        draw=black,%
        ultra thick,%
        fill=yellow,%
        decoration=zigzag,%
        decorate,%
        font=\Huge,%
        text width=0.6\textwidth,%
        align=center,%
        anchor=center,%
      ] at (0,0) {
        \begin{itemize}
          \item ポップアップ
        \end{itemize}
      };
    \end{tikzpicture}
  }

  % transitions
  % \transdissolve[duration=1]<1>
  \transduration<1>{3}%
  % \transboxout<2>%
  % \transblindsvertical<2>%
  \transwipe<2>[direction=270,duration=1]

  % \transduration<2>{2}%
  \note<1>{
    ここにノート2を入力
  }
  \note<2>{
    ここにノート2を入力
  }
\end{frame}


\begin{frame}[fragile]

  \begin{easylist}[itemize]
    @ hoge
    @@ fuga
    @@@ boke
  \end{easylist}

  \note{some stupid notes}
\end{frame}


\begin{frame}
  \frametitle{フレーム5の題名}

  \begin{columns}
    \begin{column}{0.667\linewidth}
    \tikz[remember picture] \node[] (fig1) {\includegraphics[width=\linewidth]{example-image}};
    \end{column}
    \begin{column}{0.333\linewidth}
      \begin{itemize}
        \item ここにテキストを入力
      \end{itemize}
    \end{column}
  \end{columns}

  \onslide<2>{%
    \begin{tikzpicture}[overlay,remember picture]
      \coordinate[left=1em] (node0) at (fig1.east);
      \node[%
        rectangle,%
        draw=black,%
        fill=green,
        ultra thick,%
        text width=0.3\linewidth,%
        right=0.4\linewidth,
        ] (node1) at (fig1) {%
          ポップアップ%
        };
        \draw[->,green] (node1.west)--(node0);
    \end{tikzpicture}
  }

  \onslide<3>{%
    \begin{tikzpicture}[overlay,remember picture]
      \node[%
        rectangle callout,%
        draw=black,%
        fill=white,
        ultra thick,%
        right=0.4\linewidth,%
        above=5em,%
        text width=0.4\linewidth,rounded corners=2pt,%
        callout absolute pointer=(node0),%
        ] (node2) at (fig1) {%
          ポップアップ%
        };
    \end{tikzpicture}
  }

  \iffalse
      fill=yellow,%
      decoration=zigzag,%
      decorate,%
      align=center,%
      anchor=center,%
  \fi

  % transitions
  % \transdissolve[duration=1]<1>
  \transduration<1>{2}%
  % \transboxout<2>%
  % \transblindsvertical<2>%
  \transwipe<2>[direction=270,duration=1]

  % \transduration<2>{2}%
  \note<1>{
    ここにノート2を入力
  }
  \note<2>{
    ここにノート2を入力
  }
\end{frame}
