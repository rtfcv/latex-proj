\begin{frame}
  \frametitle{フレームの題名}
  \begin{itemize}
    \item ここにテキストを入力
    \item \textcite{_article}
    \item \textcite{_book}
    \item \textcite{_incollection}
    \item \textcite{_inproceedings,_misc,_techreport}
  \end{itemize}
  \note{
    ここにノートを入力
  }
\end{frame}


\begin{frame}
  \frametitle{フレーム2の題名}
  \begin{columns}
    \begin{column}{0.5\linewidth}
      \begin{align}
        e^{i\pi} + 1 = 0
      \end{align}
    \end{column}
    \begin{column}{0.5\linewidth}
      \begin{itemize}
        \item ここにテキストを入力
      \end{itemize}
    \end{column}
  \end{columns}

  \transduration<1>{2}%
  \transfade<1>%

  \note{
    ここにノート2を入力
  }
\end{frame}


\begin{frame}
  \frametitle{フレーム3の題名}
  \begin{columns}
    \begin{column}{0.667\linewidth}
      \includegraphics[width=\linewidth]{example-image}
    \end{column}
    \begin{column}{0.333\linewidth}
      \begin{itemize}
        \item ここにテキストを入力
      \end{itemize}
    \end{column}
  \end{columns}

  \onslide<2>{%
    \begin{tikzpicture}[overlay,remember picture]
      \pgftransformshift{\pgfpointanchor{current page}{center}}%
      \node[%
        rectangle,%
        draw=black,%
        ultra thick,%
        fill=yellow,%
        decoration=zigzag,%
        decorate,%
        font=\Huge,%
        text width=0.6\textwidth,%
        align=center,%
        anchor=center,%
      ] at (0,0) {
        \begin{itemize}
          \item ポップアップ
        \end{itemize}
      };
    \end{tikzpicture}
  }%

  % transitions
  % \transdissolve[duration=1]<1>
  \transduration<1>{2}%
  % \transboxout<2>%
  % \transblindsvertical<2>%
  \transwipe<2>[direction=270,duration=1]

  % \transduration<2>{2}%

\end{frame}
\note{
  ここにノート2を入力
}
